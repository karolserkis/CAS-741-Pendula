\documentclass[12pt, titlepage]{article}

\usepackage{hyperref}
\usepackage{amsmath, mathtools}
\usepackage{amsfonts}
\usepackage{amssymb}
\usepackage{colortbl}
\usepackage{xr}
\usepackage{longtable}
\usepackage{xfrac}
\usepackage{siunitx}
\usepackage{caption}
\usepackage{pdflscape}
\usepackage{afterpage}
\usepackage{tabu}
\usepackage{verbatim}
\usepackage{url}
\usepackage{tikz}
\usepackage{enumitem}
\usepackage{extarrows}
\usepackage{graphicx}
\usepackage{tabularx}
\usepackage{float}
\usepackage{siunitx}
\usepackage{booktabs}

\usetikzlibrary{shapes.geometric, arrows}
\captionsetup{belowskip=12pt,aboveskip=4pt}

\hypersetup{
    colorlinks,
    citecolor=black,
    filecolor=black,
    linkcolor=red,
    urlcolor=blue
}
\usepackage[round]{natbib}

%% Comments

\usepackage{color}

\newif\ifcomments\commentsfalse

\ifcomments
\newcommand{\authornote}[3]{\textcolor{#1}{[#3 ---#2]}}
\newcommand{\todo}[1]{\textcolor{red}{[TODO: #1]}}
\else
\newcommand{\authornote}[3]{}
\newcommand{\todo}[1]{}
\fi

\newcommand{\wss}[1]{\authornote{blue}{SS}{#1}}
\newcommand{\spc}[1]{\authornote{magenta}{SP}{#1}}
%% Common Parts

\newcommand{\progname}{MPS } % PUT YOUR PROGRAM NAME HERE


\newcommand{\sskip}{\vskip 1mm}

% For easy change of table widths
\newcommand{\colZwidth}{1.0\textwidth}
\newcommand{\colAwidth}{0.13\textwidth}
\newcommand{\colBwidth}{0.82\textwidth}
\newcommand{\colCwidth}{0.1\textwidth}
\newcommand{\colDwidth}{0.05\textwidth}
\newcommand{\colEwidth}{0.8\textwidth}
\newcommand{\colFwidth}{0.17\textwidth}
\newcommand{\colGwidth}{0.5\textwidth}
\newcommand{\colHwidth}{0.28\textwidth}

% Used so that cross-references have a meaningful prefix
\newcounter{defnum} %Definition Number
\newcommand{\dthedefnum}{GD\thedefnum}
\newcommand{\dref}[1]{GD\ref{#1}}
\newcounter{datadefnum} %Datadefinition Number
\newcommand{\ddthedatadefnum}{DD\thedatadefnum}
\newcommand{\ddref}[1]{DD\ref{#1}}
\newcounter{theorynum} %Theory Number
\newcommand{\tthetheorynum}{T\thetheorynum}
\newcommand{\tref}[1]{T\ref{#1}}
\newcounter{tablenum} %Table Number
\newcommand{\tbthetablenum}{T\thetablenum}
\newcommand{\tbref}[1]{TB\ref{#1}}
\newcounter{assumpnum} %Assumption Number
\newcommand{\atheassumpnum}{P\theassumpnum}
\newcommand{\aref}[1]{A\ref{#1}}
\newcounter{goalnum} %Goal Number
\newcommand{\gthegoalnum}{P\thegoalnum}
\newcommand{\gsref}[1]{GS\ref{#1}}
\newcounter{instnum} %Instance Number
\newcommand{\itheinstnum}{IM\theinstnum}
\newcommand{\iref}[1]{IM\ref{#1}}
\newcounter{reqnum} %Requirement Number
\newcommand{\rthereqnum}{P\thereqnum}
\newcommand{\rref}[1]{R\ref{#1}}
\newcounter{nfreqnum} %NF Requirement Number
\newcommand{\rthenfreqnum}{P\thenfreqnum}
\newcommand{\nfref}[1]{NF\ref{#1}}
\newcounter{lcnum} %Likely change number
\newcommand{\lthelcnum}{LC\thelcnum}
\newcommand{\lcref}[1]{LC\ref{#1}}
\newcommand{\sref}[1]{\S~\ref{#1}}

\usepackage{fullpage}

\newcounter{acnum}
\newcommand{\actheacnum}{AC\theacnum}
\newcommand{\acref}[1]{AC\ref{#1}}

\newcounter{ucnum}
\newcommand{\uctheucnum}{UC\theucnum}
\newcommand{\uref}[1]{UC\ref{#1}}

\newcounter{mnum}
\newcommand{\mthemnum}{M\themnum}
\newcommand{\mref}[1]{M\ref{#1}}

\begin{document}

\title{CAS 741: Unit Verification and Validation Plan\\[10pt]\Large Dynamical Systems: \progname}
\author{Karol Serkis\\\texttt{serkiskj@mcmaster.ca}\\GitHub:
\href{https://www.github.com/karolserkis}{karolserkis}}
\date{\today}
	
\maketitle

\pagenumbering{roman}

\section{Revision History}

\begin{tabularx}{\textwidth}{p{4cm}p{2cm}X}
\toprule {\bf Date} & {\bf Version} & {\bf Notes}\\
\midrule
December 5, 2018 & 1.0 &  First full draft for submission\\
\bottomrule
\end{tabularx}

~\newpage

\section{Symbols, Abbreviations and Acronyms}

See SRS Documentation at 
\href{https://github.com/karolserkis/CAS-741-Pendula/blob/master/docs/SRS/SRS.pdf}{CAS-741-Pendula SRS}
\wss{give url}

\wss{Also add any additional symbols, abbreviations or acronyms}\\
The symbols are listed in alphabetical order.\\

\renewcommand{\arraystretch}{1.2}
\begin{tabular}{l l} 
  \toprule		
  \textbf{symbol} & \textbf{description}\\
  \midrule 
  A & Assumption\\
  DD & Data Definition\\
  FT & Functional Test \\
  GD & General Definition\\
  GS & Goal Statement\\
  IM & Instance Model\\
  LC & Likely Change\\
  MG & Module Guide\\ 
  MIS & Module Interface Specification\\ 
  NF & Non-Functional Requirement\\
  R & Requirement\\
  SRS & Software Requirements Specification\\
  T & Test\\
  VnV & Verification and Validation\\ 
  \bottomrule
\end{tabular}\\

\subsection{Table of Units}

Throughout this document SI (Syst\`{e}me International d'Unit\'{e}s) is
employedas the unit system. In addition to the basic units, several derived
units are
used as described below.  For each unit, the symbol is given followed by a
description of the unit and the SI name.\\

\renewcommand{\arraystretch}{1.2}
\begin{table}[h!]
	\centering
\begin{center}
  \noindent \begin{tabular}{l l l} 
    \toprule		
    \textbf{symbol} & \textbf{unit} & \textbf{SI}\\
    \midrule 
    \si{\metre} & length & metre\\
    \si{\kilogram} & mass & kilogram\\
    \si{\second} & time & second\\
    \si{\degree} & angle & degree\\
    \bottomrule
  \end{tabular}
\end{center}
	\caption{Table of Units}
	\label{Table:R_trace}
\end{table}

\newpage

\tableofcontents

\listoftables

%\listoffigures

\newpage

\pagenumbering{arabic}

The purpose of the document is to provide the Unit Verification and Validation Plan for testing the \progname{}software with 
respect to the requirements (see SRS document). Unit Verification and 
Validation plan consists of outlining test cases for particular units of a 
software's modules. These tests are created to ensure that the units satisfy 
the software's functional and nonfunctional requirements. The tests can be 
traced to a particular module. The module should be traced to a particular 
requirement. 

\section{General Information}

This documents is an Unit VnV Plan for the \progname program. The
directory for this project can be found at GitHub:
\href{https://github.com/karolserkis/CAS-741-Pendula/}{/karolserkis/CAS-741-Pendula/}\\

\subsection{Purpose}

The purpose of this document is to describe the unit VnV plan and requirements for the
\progname program solution that
only focuses on multi-pendulum simulations (double \& triple pendula and beyond) and tracking the chaotic
motion of the system. It will allow users to generate diagrams (e.g. Poincare
mapping)
and plot trajectories over time using two different ODE/DAE initial value
problem solvers. In the case of
a double pendulum you have a new system that is dynamic and chaotic and
requires a set of coupled ordinary differential equation solvers. Once one
introduces
multiple
pendula the system becomes chaotic and interesting to model and simulate. 

\subsection{Scope}

The scope of the \progname program is limited to the generation 
of diagrams and plot trajectories that are possible to run and compute on a
local system.

The scope of the test plan is described below:
\begin{itemize}
\item \progname{} will be written in Python.
\item The proposed test plan is focusing on system and unit testing to
  verify the functional and non functional requirements of
  \progname{}
\end{itemize}

The \progname is limited to the user
initialized inputs and the output of the \progname will either plot trajectories over time, generate
diagrams, like Poincare mapping
and limit the user to a specific duration of the simulation, in order to allow
diagrams and trajectory history to be saved.
The user will be able to set a range of time and initialize the system. \\

The modules that might be tested for under the Unit VnV plan that are traced to the SRS requirements are:
\begin{description}
\item [\refstepcounter{mnum} \mthemnum \label{mHH}:] Hardware-Hiding Module (described in MG$\_$Hardware)
\item [\refstepcounter{mnum} \mthemnum \label{mHH}:] \progname Control Module (described in MG$\_$Control)
\item [\refstepcounter{mnum} \mthemnum \label{mHH}:] \progname GUI Module (described MG$\_$GUI)
\item [\refstepcounter{mnum} \mthemnum \label{mHH}:] User Input Parameters Module (described in MG$\_$InputFormat)
\item [\refstepcounter{mnum} \mthemnum \label{mHH}:] Lagrangian Module (described in MG$\_$LA)
\item [\refstepcounter{mnum} \mthemnum \label{mHH}:] Hamiltonian Module (described in MG$\_$HA)
\item [\refstepcounter{mnum} \mthemnum \label{mHH}:] Data Structure Module (described in MG$\_$DataStruct)
\item [\refstepcounter{mnum} \mthemnum \label{mHH}:] Generic GUI/Plot Module (described in MG$\_$GUIplot)
\item [\refstepcounter{mnum} \mthemnum \label{mHH}:] Generic Trajectory Simulation GUI Module (described in MG$\_$SIM)
\end{description}

%\subsection{Overview of Document}

\section{Plan}
	
%\subsection{Software Description}

\subsection{Test Team}

The Verification and Validation team consists of:\\
Karol Serkis
\wss{Probably just you.  :-)}

\subsection{Automated Testing Approach}

Unit-based scripts can be created for testing the modules in \progname{}and be tested automatically with 
scripts written in pytest. See section on verification tools for more information.

\subsection{Verification Tools}

Python has a few open source unit testing frameworks, namely, \\pytest: \url{https://docs.pytest.org}. 
Unit-based scripts can be created for testing the modules in \progname. The functions and classes that automate 
this process are detailed in the following link: 
\url{https://github.com/pytest-dev/pytest}.\\
\wss{Thoughts on what tools to use, such as the following: unit testing
  framework, valgrind, static analyzer, make, continuous integration, test
  coverage tool, etc.}

% \subsection{Testing Schedule}
		
% See Gantt Chart at the following url ...

\subsection{Non-Testing Based Verification}

The non-testing based verfication will involve a code inspection by myself and
Prof Smith during final submission. He will inspect each of the modules to guarentee that the 
physical equations have been implemented succesfully, verify that the software 
is designed to be maintainable and manageable and complete the 
two surveys in the appendix.\\
\wss{List any approaches like code inspection, code walkthrough, symbolic
  execution etc.  Enter not applicable if that is the case.}

\section{Unit Test Description}

The modules that are being unit tested are specificed in the MIS 
(\url{https://github.com/karolserkis/CAS-741-Pendula/blob/master/docs/}). The unit VnV plan 
ensure that each module is behaving accurately and that each module is 
satisfying the SRS requirements.\\ 
	
\subsection{Tests for Functional Requirements}

\subsubsection{User Input Module}

The following tests were created to ensure that the input parameters module is 
storing the correct values specified by the user. This first step is critical 
as all of the other modules rely on these quantities.

\begin{enumerate}				
\item{test-InParams\\}

Type: Automaic

Initial State: 

Input: Input from Table of Units:\\
  \noindent \begin{tabular}{l l l} 
    \toprule		
    \textbf{symbol} & \textbf{unit} & \textbf{SI}\\
    \midrule 
    \si{\metre} & length & metre\\
    \si{\kilogram} & mass & kilogram\\
    \si{\second} & time & second\\
    \si{\degree} & angle & degree\\
    \bottomrule
  \end{tabular}

Output: assert=True

Test Case Derivation: The environment variable inputted by the user should 
match the state variable meant to store the value.\\
Ex. units for length, mass and time cannot be negative.

How test will be performed: Integer numbers will be inputed in the environment. An assert statement will return true if these 
are equal.
    
\end{enumerate}

\subsubsection{\progname Lagrangian Module Testing}

The following tests were created to ensure that the input parameters module is 
storing the correct values specified by the user. This first step is critical 
as all of the other modules rely on these quantities.

\begin{enumerate}				
	\item{test-Lagrangian\\}
	
	Type: Automaic
	
	Initial State: 
	
	Input: Input from Table of Units:\\
  \noindent \begin{tabular}{l l l} 
    \toprule		
    \textbf{symbol} & \textbf{unit} & \textbf{SI}\\
    \midrule 
    \si{\metre} & length & metre\\
    \si{\kilogram} & mass & kilogram\\
    \si{\second} & time & second\\
    \si{\degree} & angle & degree\\
    \bottomrule
  \end{tabular}
	
	Output: assert=True
	
	Test Case Derivation: From instance model of the SRS it follows that:
	Pendulum Lagrangian ($L=T-V$)
	(Refer to SRS for equations)
	
	How test will be performed: We will check to see if each pendula and plot behaves as expected.
	Testing the Kinetic Energy equation:
	$$ T = \displaystyle\frac{1}{2}m_1v_1^2 + \frac{1}{2}m_2v_2^2 $$
$$ = \frac{1}{2}m_1(\dot{x}_1^2 + \dot{y}_1^2) + \frac{1}{2}m_2(\dot{x}_2^2 +
\dot{y}_2^2) $$
$$ = \frac{1}{2}m_1 l_1^2 \dot{\theta}_1^2 + \frac{1}{2}m_2\left[l_1^2
\dot{\theta}_1^2 + l_2^2 \dot{\theta}_2^2 + 2l_1l_2\dot{\theta}_1\dot{\theta}_2
\cos(\theta_1 - \theta_2)\right]$$
			
\end{enumerate} 

\subsubsection{\progname Hamiltonian Module Testing}

The following tests were created to ensure that the input parameters module is 
storing the correct values specified by the user. This first step is critical 
as all of the other modules rely on these quantities.

\begin{enumerate}				
	\item{test-Hamiltonian\\}
	
	Type: Automaic
	
	Initial State: 
	
	Input: Input from Table of Units:\\
  \noindent \begin{tabular}{l l l} 
    \toprule		
    \textbf{symbol} & \textbf{unit} & \textbf{SI}\\
    \midrule 
    \si{\metre} & length & metre\\
    \si{\kilogram} & mass & kilogram\\
    \si{\second} & time & second\\
    \si{\degree} & angle & degree\\
    \bottomrule
  \end{tabular}
	
	Output: assert=True
	
	Test Case Derivation: From instance model of the SRS it follows that:
	(Refer to SRS for equations)
	
	How test will be performed: We will check to see if each pendula and plot behaves as expected.
	Testing the Kinetic Energy equation:
	$$ T = \displaystyle\frac{1}{2}m_1v_1^2 + \frac{1}{2}m_2v_2^2 $$
$$ = \frac{1}{2}m_1(\dot{x}_1^2 + \dot{y}_1^2) + \frac{1}{2}m_2(\dot{x}_2^2 +
\dot{y}_2^2) $$
$$ = \frac{1}{2}m_1 l_1^2 \dot{\theta}_1^2 + \frac{1}{2}m_2\left[l_1^2
\dot{\theta}_1^2 + l_2^2 \dot{\theta}_2^2 + 2l_1l_2\dot{\theta}_1\dot{\theta}_2
\cos(\theta_1 - \theta_2)\right]$$
			
\end{enumerate} 
		
\subsubsection{Plotting Module} 

\begin{enumerate}				
	\item{test-Plotting-Poincare\'{e}\\}
	
	Type: Manual
	
	Initial State: 
	
	Input: User Input, Data structure\\
	Input from Table of Units:\\
  \noindent \begin{tabular}{l l l} 
    \toprule		
    \textbf{symbol} & \textbf{unit} & \textbf{SI}\\
    \midrule 
    \si{\metre} & length & metre\\
    \si{\kilogram} & mass & kilogram\\
    \si{\second} & time & second\\
    \si{\degree} & angle & degree\\
    \bottomrule
  \end{tabular}
	
	Output: Time pendulum trail plot and Poincare\'{e} map
	
	Test Case Derivation: The Lagrangian Module, Hamiltonian Module and Data structure module should produce valid
	output that corresponds to the reference material and examples.
	
	How test will be performed: I will inspect the plots to see if 
	they resmble the examples in my references. See references [1] and [7]
\end{enumerate} 

\begin{enumerate}				
	\item{test-Plotting-Trajectory\\}
	
	Type: Manual
	
	Initial State: 
	
	Input: User Input, Data structure
$$\sum \mathbb{R} (\texttt{for Langrangian equation})$$
$$ P(x) :\mathbb{Z} \times \mathbb{R} \implies \mathbb{R}$$\\
	
	Output: Trajectory plot of multiple pendula
	
	Test Case Derivation: The Lagrangian Module and Data structure module should produce valid
	output that corresponds to the reference material and examples.
	
	How test will be performed: I will inspect the plots to see if 
	they resmble the examples in my references. See references [1],[2],[3],[4],[5] and [6].
\end{enumerate}

\subsubsection{GUI/Plot Module} 

\begin{enumerate}				
	\item{test-GUI-Plot-Simulation\\}
	
	Type: Manual
	
	Initial State: 
	
	Input: User Input, Data structure\\
	Input from Table of Units:\\
  \noindent \begin{tabular}{l l l} 
    \toprule		
    \textbf{symbol} & \textbf{unit} & \textbf{SI}\\
    \midrule 
    \si{\metre} & length & metre\\
    \si{\kilogram} & mass & kilogram\\
    \si{\second} & time & second\\
    \si{\degree} & angle & degree\\
    \bottomrule
  \end{tabular}
	
	Output: GUI interface and pendulum trail plot with axis
	
	Test Case Derivation: The Lagrangian Module, Hamiltonian Module and Data structure module should produce valid
	output that corresponds to the reference material and examples.
	
	Check GUI for visual integrity
	
	How test will be performed: I will inspect the plots to see if 
	they resmble the examples in my references. See references [1],[2],[3],[4],[5] and [6].
\end{enumerate} 

\subsection{Tests for Nonfunctional Requirements}

Planning for nonfunctional tests of units will not be relevant to \progname. 
For system tests related to Nonfunctional requirements please see the System 
Verification and Validation Plan at 
\url{https://github.com/karolserkis/CAS-741-Pendula/blob/master/docs/}. 

\subsection{Traceability Between Test Cases and Requirements}

\begin{table}[H]
	\centering
	\label{Table:trace}
	\begin{tabular}{|c|c|c|c|c|c|c|}
		\hline        
		& \tref{lagrangian}
		& \tref{kinetic}
		& \tref{potential}
		& \tref{poincare}
		& \ddref{real-interv}
		& \iref{add-real} \\
		\hline
        \tref{lagrangian} & X & X & X & X & X & X \\ \hline
        \tref{kinetic} & X & X & X & X & X & X\\ \hline
        \tref{potential} & X & X & X & X & X & X\\ \hline
        \tref{poincare} &  &  &  &  & X & X\\ \hline
        \ddref{real-interv} & X & X & X &  & X & X\\ \hline
        \iref{add-real} & X & X & X &  & X & X\\ \hline
	\end{tabular}
	\caption{Traceability Matrix Showing the Connections Between Items of 
	Different Sections}
\end{table}


\begin{table}[h!]
	\centering
	\begin{tabular}{|c|c|c|c|c|c|c|}
		\hline        
		& \tref{lagrangian}
		& \tref{kinetic}
		& \tref{potential}
		& \tref{poincare}
		& \ddref{real-interv}
		& \iref{add-real} \\
		\hline
        \rref{funinput} & X & X & X &  & X & X \\ \hline
        \rref{funkinpot} & X & X & X & X & X & X\\ \hline
        \rref{funlagham} & X & X & X &  & X & X \\ \hline
        \rref{funplot} & X & X & X & X & X & X\\ \hline
	\end{tabular}
	\caption{Traceability Matrix Showing the Connections Between Requirements 
	and Instance Models}
	\label{Table:R_trace}
\end{table}

% \section{Tests for Proof of Concept}

% \subsection{Area of Testing1}
		
% \paragraph{Title for Test}

% \begin{enumerate}

% \item{test-id1\\}

% Type: Functional, Dynamic, Manual, Static etc.
					
% Initial State: 
					
% Input: 
					
% Output: 
					
% How test will be performed: 
					
% \item{test-id2\\}

% Type: Functional, Dynamic, Manual, Static etc.
					
% Initial State: 
					
% Input: 
					
% Output: 
					
% How test will be performed: 

% \end{enumerate}

% \subsection{Area of Testing2}

% ...
				
\section{Unit Testing Plan}
		
\wss{Unit testing plans for internal functions and, if appropriate, output
  files}

\bibliographystyle{plainnat}

\bibliography{SRS}

\newpage

\section{Appendix}

This is where you can place additional information.

\subsection{Symbolic Parameters}

The definition of the test cases will call for SYMBOLIC\_CONSTANTS.
Their values are defined in this section for easy maintenance.

\subsection{Software Verification Checklist} 
\label{softwarevercheck}
\begin{itemize}
	\item Did any of the inputs you entered provide suprising results? If yes, 
	what were they?
	\item Were all of the plots legible? 
\end{itemize} 

\subsection{Usability Survey Questions?}
\begin{enumerate}
	\item Were you able to setup and execute the program using the comment
	and makefile instructions alone, or did you require additional troubleshooting/help? 
	If any steps were unclear, please explain how they might be improved.
	\item Did the program accept your inital input?
	\item Did the program output the trajectory plot simulation?
	\item Did the program run the simulation to its completion and smoothly without issues?
	\item Did the plot of pendulum movement pattern and/or Poincar\'{e} map?
\end{enumerate}

\subsection*{References}\label{ssec:ref}
\begin{itemize}
\item{[1]} Dynamics of multiple pendula \\\url{http://wmii.uwm.edu.pl/~doliwa/IS-2012/Szuminski-2012-Olsztyn.pdf}
\item{[2]} Pendulum \\\url{https://en.wikipedia.org/wiki/Pendulum}
\item{[3]} Pendulum (mathematics)
\\\url{https://en.wikipedia.org/wiki/Pendulum_(mathematics)}
\item{[4]} Double Pendulum
\\\url{https://en.wikipedia.org/wiki/Double_pendulum}\item{[4]}
Differential-Algebraic Equations by Taylor Series
\\\url{http://www.cas.mcmaster.ca/~nedialk/daets/}
\item{[5]} Multi-body Lagrangian Simulations
\\\url{https://www.youtube.com/channel/UCCuLchOx0W0yoNE9KOCYlVQ}
\item{[6]} The double pendulum: Lagrangian formulation
\\\url{https://diego.assencio.com/?index=1500c66ae7ab27bb0106467c68feebc6}
\item{[7]} Poincaré map
\\\url{https://en.wikipedia.org/wiki/Poincar%C3%A9_map}
\item{[8]} D. L. Parnas, ``On the criteria to be used in decomposing systems into modules'', Comm.
ACM, vol. 15, pp. 1053-1058, December 1972.
\item{[9]} D. Parnas, P. Clement, and D. M. Weiss, ``The modular structure of complex systems'',
in International Conference on Software Engineering, pp. 408-419, 1984.
\item{[10]} D. L. Parnas, ``Designing software for ease of extension and contraction,'' in ICSE '78:
Proceedings of the 3rd international conference on Software engineering, \\
(Piscataway, NJ, USA), pp. 264-277, IEEE Press, 1978.
\item{[11]} Smith and Lai(2005); Smith et al. (2007). \\
See bib file that doesn't work for me for full citation.
\end{itemize}

\end{document}
