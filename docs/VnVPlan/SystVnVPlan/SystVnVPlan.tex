\documentclass[12pt, titlepage]{article}

\usepackage{hyperref}
\usepackage{amsmath, mathtools}
\usepackage{amsfonts}
\usepackage{amssymb}
\usepackage{graphicx}
\usepackage{colortbl}
\usepackage{xr}
\usepackage{longtable}
\usepackage{xfrac}
\usepackage{tabularx}
\usepackage{float}
\usepackage{siunitx}
\usepackage{caption}
\usepackage{pdflscape}
\usepackage{afterpage}
\usepackage{tabu}
\usepackage{verbatim}
\usepackage{url}
\usepackage{tikz}
\usepackage{enumitem}
\usepackage{extarrows}
\usepackage{booktabs}
\usepackage[round]{natbib}
\usetikzlibrary{shapes.geometric, arrows}

\captionsetup{belowskip=12pt,aboveskip=4pt}

\makeatletter
\newcommand*\bigcdot{\mathpalette\bigcdot@{.7}}
\newcommand*\bigcdot@[2]
  {\mathbin{\vcenter{\hbox{\scalebox{#2}{$\m@th#1\bullet$}}}}}
\makeatother

%\usepackage{refcheck}

\hypersetup{
    bookmarks=true,         % show bookmarks bar?
    colorlinks=true,        % false: boxed links; true: colored links
    linkcolor=red,          % color of internal links 
                            %  (change box color with linkbordercolor)
    citecolor=blue,         % color of links to bibliography
    filecolor=magenta,      % color of file links
    urlcolor=blue           % color of external links
}

\usepackage{fullpage}

%% Comments

\usepackage{color}

\newif\ifcomments\commentsfalse

\ifcomments
\newcommand{\authornote}[3]{\textcolor{#1}{[#3 ---#2]}}
\newcommand{\todo}[1]{\textcolor{red}{[TODO: #1]}}
\else
\newcommand{\authornote}[3]{}
\newcommand{\todo}[1]{}
\fi

\newcommand{\wss}[1]{\authornote{blue}{SS}{#1}}
\newcommand{\spc}[1]{\authornote{magenta}{SP}{#1}}
%% Common Parts

\newcommand{\progname}{MPS } % PUT YOUR PROGRAM NAME HERE


\begin{document}
\title{CAS 741: System Verification and Validation Plan\\[10pt]
\Large Dynamical Systems: \progname}
\author{Karol Serkis\\\texttt{serkiskj@mcmaster.ca}\\GitHub:
\href{https://www.github.com/karolserkis}{karolserkis}}
\date{\today}
	
\maketitle

\pagenumbering{roman}

\section{Revision History}

\begin{tabularx}{\textwidth}{p{3.3cm}p{2.5cm}X}
\toprule {\bf Date} & {\bf Developer(s)} & {\bf Notes}\\
\midrule
October 22, 2018 & Karol Serkis & First revision of document\\
October 21, 2018 & Karol Serkis & New Blank Project Template utilized \\
Dec. 17, 2018 & Karol Serkis & All GitHub issues and comments addressed \\
Dec. 18, 2018 & Karol Serkis & SRS \& bibliography fixed, 
\& program name fixed \\
\bottomrule
\end{tabularx}

~\newpage

\section{Symbols, Abbreviations and Acronyms}

\renewcommand{\arraystretch}{1.2}
\begin{tabular}{l l} 
  \toprule		
  \textbf{symbol} & \textbf{description}\\
  \midrule 
  T & Test\\
  R & Requirement\\ 
  NFR & Non-functional Requirement\\ 
  \bottomrule
\end{tabular}\\

\subsection{Table of Units}

Throughout this document SI (Syst\`{e}me International d'Unit\'{e}s) is
employed as the unit system. In addition to the basic units, several derived
units are
used as described below.  For each unit, the symbol is given followed by a
description of the unit and the SI name.\\

\renewcommand{\arraystretch}{1.2}
  \noindent \begin{tabular}{l l l} 
    \toprule		
    \textbf{symbol} & \textbf{unit} & \textbf{SI}\\
    \midrule 
    \si{\metre} & length & metre\\
    \si{\kilogram} & mass & kilogram\\
    \si{\second} & time & second\\
    \si{\degree} & angle & degree\\
    \bottomrule
  \end{tabular}

\subsection{Abbreviations and Acronyms}

The symbols are listed in alphabetical order.\\

\renewcommand{\arraystretch}{1.2}
\begin{tabular}{l l} 
  \toprule		
  \textbf{symbol} & \textbf{description}\\
  \midrule 
  A & Assumption\\
  DD & Data Definition\\
  GS & Goal Statement\\
  IM & Instance Model\\
  LC & Likely Change\\
  NF & Non-Functional Requirement\\
  PS & Physical System Description\\
  R & Requirement\\
  SRS & Software Requirements Specification\\
  T & Theoretical Model\\
  \bottomrule
\end{tabular}\\

\tableofcontents

\listoftables

%\listoffigures

\newpage

\pagenumbering{arabic}

The purpose of this document is to provide a comprehensive plan for testing the 
\progname software against the requirements described in the
\progname SRS. 
Complementary documents include the System Requirement Specifications, Module
Guide and Module Interface Specification.  The full documentation and
implementation can be found at:
\url{https://github.com/karolserkis/CAS-741-Pendula}. 
After reading this document one should be able to create and run test cases to 
verify and validate \progname
\wss{provide an introductory blurb and road map of the Verification and 
Validation plan}

\section{General Information}

\subsection{Summary}

This document discusses the verification and validation requirements and
provides a detailed description of the testing that will be carried out on the
\progname  program.
The \progname will allow users to generate trajectory simulations and 
plots of kinetic and potential energy, over time using two different ODE/DAE initial value
problem solvers. See the instance models (IM) in the SRS for more details.

\wss{Say what software is being tested.  Give its name and a brief overview of
  its general functions.}

\subsection{Objectives}
The purpose of this section is to describe the objectives for the
\progname program solution that
only focuses on multi-pendulum simulations and tracking the chaotic
motion of the system. It will allow users to generate trajectory simulations
and plots of kinetic and potential energy over time using two different ODE/DAE initial 
value problem solvers. The implementation should be reliable and accurate.\\\\
The System VnV plan will in summary:
\begin{itemize}
\item Satisfy the requirements of Dr.\ Spencer Smith and those
  outlined in the SRS.
\item Build confidence in the software correctness.
\item Verify usability and effectiveness.
\item The tests outlined in this document are limited to dynamic tests only.  
No formal static testing (code walk-through, code inspections, etc.) will be 
carried out.
\item The \progname software will be written in Python. The testing of 
implementations in other languages will not be considered in this document.
\end{itemize}

\wss{State what is intended to be accomplished.  The objective will be around
  the qualities that are most important for your project.  You might have
  something like: ``build confidence in the software correctness,''
  ``demonstrate adequate usability.'' etc. You won't list all of the qualities,
  just those that are most important.}

\subsection{Relevant Documentation}
SRS document:
\begin{itemize}
	\item Karol Serkis. System Requirements Specification for \progname. 
	Github, 2018.
\end{itemize}
\wss{Reference relevant documentation. This will definitely include your SRS}

\wss{A hyper-link to your GitHub documentation would be great.}

\section{Plan}
	
\subsection{Verification and Validation Team}
The test team includes the following members: Karol Serkis

The verification and validation team consists of (possibly Dr. Spencer Smith) 
classmates and I. \\

\wss{Probably just you.  :-)}

\subsection{SRS Verification Plan}

The SRS verification plan consists of feedback from Dr. Spencer Smith and 
CAS 741 classmates. 
They will provide feedback regarding model assumptions, constraints and goals. 
Feedback from classmates and Dr. Smith will criticize the document outline, 
readability and requirements. \\ \wss{Using Repos.xlsx you can be more specific
  about how is going to provide you with feedback.}  \wss{You could mention the
  use of the GitHub issue tracker.}

\wss{List any approaches you intend to use for SRS verification. This may just
  be ad hoc feedback from reviewers, like your classmates, or you may have
  something more rigorous/systematic in mind..}

\subsection{Design Verification Plan}

The design verification plan will simply involve inspection of the software by 
Dr. Spencer Smith and CAS 741 classmates. \\ \wss{Same comment as above.}

\wss{Plans for design verification}

\subsection{Implementation Verification Plan}

The implementation verification plan consists of two parts. The first part 
is a software verification. It will be completed by myself any others who 
use \progname and will involve verification that basic that basic software 
features have been implemented successfully, 
such as allowing the user to successfully utilize the software to full-fill
its basic responsibilities (ie generate simulations and plot trajectories). 
More can be found in the appendix. The 
second part involves running software tests 
outlined in sections 5 (System Test) and 6 (Static Verification). 
Unit testing will also be performed. 

\wss{You should at least point to the tests listed in this document 
and the unit testing plan.}

\subsection{Software Validation Plan}

See SRS Documentation at 
\href{https://github.com/karolserkis/CAS-741-Pendula/blob/master/docs/SRS/SRS.pdf}
{CAS-741-Pendula SRS}

\wss{If there is any external data that can be used for validation, you should
  point to it here.  If there are no plans for validation, you should state 
  that here.}

\section{System Test Description}
	
\subsection{Tests for Functional Requirements}

\wss{Subsets of the tests may be in related, so this section is divided into
  different areas.  If there are no identifiable subsets for the tests, this
  level of document structure can be removed.}
\begin{enumerate}
\item{test-Rin1}	
				
Initial State: -
					
Input: Initial input for the pendula weights in chain. See SRS and appendix for 
more info.
					
Output: Error or pass message. (how many pendulums did you initialize?)
(did you provide a positive integer > 1 and < 6?
					
How test will be performed: Combinations of inputting non-numerical values as 
input  (such as letters), or numerical values outside of their respective 
constraints, will be considered. A successful test in these instances will be 
an error message. \\
I will also test cases with each variable in the input having an 
acceptable numerical value. A successful test in this case will be a pass 
message. 
					
\item{test-Rplt} 

Initial State: -

Input: Similar to test-Rin1 

Output: Pass of fail message (Is there a graph/plot or not)

How test will be performed: This test will simply check to see if the 
trajectory plot or Poincare plot is plotted and if the picture agrees with the 
related physics.
There should be no plot for the failed test cases of test-Rin1. 

\item{test-Rstl} 

Initial State: -

Input: Similar to test-Rin1 

Output: Binary Variable (Verification of stability) 

How test will be performed: The stability results will be compared with the 
stability analysis in (ref1). (Also see Usability
Survey in the Appendix)

\end{enumerate}

\wss{The test cases are specific enough.  Each test case should only test one
  thing.  That means the output cannot have an ``or'' in it.  There should be
  enough information for someone to write the test without having to look
  anything up, or having to consult with you. You want to give specific values
  for the inputs.}

\subsection{Tests for Nonfunctional Requirements}

\begin{enumerate}

\item{test-NFR1\\}

Type: Static
					
Initial State: -
					
Input/Condition: \progname Python code
					
Output/Result: Pass or Fail
					
How test will be performed: The software will be manually read by the developer 
and his supervisor to see if there is a more effective code structure to allow 
implementation of plot trajectories kinetic \& potential energy plot and related ODE solvers
utilized.  \wss{Static tests like this can be very helpful, but you should try
  to be more rigorous.  Is there a checklist to follow?  Is there a process for
  conducting code walkthroughs?}

\item{test-NFR2\\} 

Type: Manual 

Initial State: Software system with prescribed input.

Input: Input all inital user values to the ODE equation/Lagrangian equation.

Output: Pass or fail (See SRS for equations).

How test will be performed: The output ought fit with the physics and math 
simulations for \progname. This test will loop through a various set of inputs 
to see if the plot does not go out of bounds or fail with overflow. (Also see Usability
Survey in the Appendix)

\item{test-NFR3\\} 

Type: Manual 

Initial State: Software system with prescribed input.

Input: See SRS and appendix for more detail.

Output: Pass or fail.

How test will be performed: The output will be tested against the Simulations
referenced in the SRS that are already available to compare. Also reference
Dr. Ned's software and other related. (Also see Usability
Survey in the Appendix) \\

\end{enumerate}

\wss{Your test cases need to be more specific.  What are the inputs for your
  tests and what are the expected outputs. For the expected outputs you need to
say how you are going to calculate the error.  Your output will likely be a
sequence of values that you need to compare to the ``correct'' sequence of
values.  You should consider using the ratio of the Euclidean norm of the
residual vector to the Euclidean norm of the expected output vector.}

\newpage
\subsection{Traceability Between Test Cases and Requirements}

\begin{table}[h]
	\centering
	\begin{tabular}{|c|c|c|c|c|c|c|c|c|c|c|c|c|c|c|c|c|c|c|c|c|c|}
		\hline        
		& Rin& Rplt & Rstl & NFR1 & NFR2 & NFR3 \\
		\hline
		test-Rin     &X & X &X  &X &X &X  \\ \hline
		test-Rplt    & X  &X & &X & X &X \\ \hline 
		test-Rstl    &X  & & & & &  \\ \hline 
		test-NFR1    & X & X& &X &X &X  \\ \hline 
		test-NFR2  &X  & X& & X&X &X \\ \hline 
		test-NFR3  &X  &X & &X & X &X \\
		\hline
	\end{tabular}\\
	\caption{Traceability Between Test }
	\label{Table:D_1}
\end{table} 

\wss{Provide a table that shows which test cases are supporting which
  requirements.}

\section{Static Verification Techniques}

\begin{itemize}
	\item Code inspection : I will 
	go through the code to see if each step is correct with respect to the
	mathematical theory. In particular I will ensure that: 
	\begin{itemize}
	\item any discretization of functions is performed accurately from the 
	pivot points. 
	\item plot trajectories are within scope
	\item ensure a variable is not accidentally 
	overwritten or cleared. 
	\item variables are being used in the right context. 
	For example, equal increment in all used Cartesian coordinates. 
	\item functions from other packages are being used in the right 
	context. For example, some packages have different standards for 
	constants.
	\end{itemize} 
	\item Code walk-through: Dr. Spencer Smith and I will go 
	through the code together to ensure that: 
	\begin{itemize}
	\item I correctly implemented the mathematical theory and numerical 
	algorithms.
	\item I made the code manageable and maintainable for future use.
	\end{itemize}
\end{itemize}

\wss{These ideas are a good starting point. I suggest that you flesh them out
  more.  Also, you can delete this section and move the contents to Section 4.4
  (Implementation Verification Plan.)  I've modified the template after reading
  several student examples.  It turns out that this section and Section 4.4 are
  really covering the same topic.}

\wss{In this section give the details of any plans for static verification of
  the implementation.  Potential techniques include code walk-through, code
  inspection, static analyzers, etc.}
				
\bibliographystyle{plainnat}

\bibliography{SRS}

\newpage

\section{Appendix}

This is where you can place additional information.

\subsection{Symbolic Parameters}

The definition of the test cases will call for SYMBOLIC\_CONSTANTS.
Their values are defined in this section for easy maintenance.

\subsection{Usability Survey Questions?}

\begin{itemize}
	\item How long did it take to learn how to run the software? 
	\item Was it easy to interpret the output? 
	\item What aspects of this software do you feel need improvement?
	\item Was the output received and processed in an 
	adequate amount of time?
	\item How does this program compare with other similar software?
\end{itemize} 

\wss{Did you reference this survey from the body of your report?  I couldn't
  find a reference.   You should connect this to a NFR test related to
  usability.  I like your questions, but try to put some time into making them
  less ambiguous.  Was it easy to interpret the output is a difficult question
  to answer.  Instead you could ask them how long it took them to interpret the
  output?  (You want to ask questions that are measurable and unambiguous.)}
\wss{This is a section that would be appropriate for some projects.}

\end{document}
