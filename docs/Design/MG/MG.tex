\documentclass[12pt, titlepage]{article}

\usepackage{amsmath, mathtools}
\usepackage{amsfonts}
\usepackage{amssymb}
\usepackage{colortbl}
\usepackage{xr}
\usepackage{longtable}
\usepackage{xfrac}
\usepackage{siunitx}
\usepackage{caption}
\usepackage{pdflscape}
\usepackage{afterpage}
\usepackage{tabu}
\usepackage{verbatim}
\usepackage{url}
\usepackage{tikz}
\usepackage{enumitem}
\usepackage{extarrows}
\usetikzlibrary{shapes.geometric, arrows}

\usepackage{fullpage}
\usepackage[round]{natbib}
\usepackage{multirow}
\usepackage{booktabs}
\usepackage{tabularx}
\usepackage{graphicx}
\usepackage{float}
\usepackage{hyperref}
\hypersetup{
    colorlinks,
    citecolor=black,
    filecolor=black,
    linkcolor=red,
    urlcolor=blue
}

\input{../../Comments}
%% Common Parts

\newcommand{\progname}{MPS } % PUT YOUR PROGRAM NAME HERE


\newcounter{acnum}
\newcommand{\actheacnum}{AC\theacnum}
\newcommand{\acref}[1]{AC\ref{#1}}

\newcounter{ucnum}
\newcommand{\uctheucnum}{UC\theucnum}
\newcommand{\uref}[1]{UC\ref{#1}}

\newcounter{mnum}
\newcommand{\mthemnum}{M\themnum}
\newcommand{\mref}[1]{M\ref{#1}}

\begin{document}

\title{CAS 741: Module Guide\\[10pt]\Large Dynamical Systems: \progname}
\author{Karol Serkis\\\texttt{serkiskj@mcmaster.ca}\\GitHub:
\href{https://www.github.com/karolserkis}{karolserkis}}
\date{\today}

\maketitle

\pagenumbering{roman}

\section{Revision History}

\begin{tabularx}{\textwidth}{p{4cm}p{2cm}X}
\toprule {\bf Date} & {\bf Version} & {\bf Notes}\\
\midrule
October 31, 2018 & 1.0 &  First full draft for submission\\
December 17, 2018 & 2.0 & All GitHub issues and comments addressed \\
December 18, 2018 & 3.0 & MG \& bibliography fixed, 
\& program name fixed \\
\bottomrule
\end{tabularx}

\newpage

\tableofcontents

\listoftables

\listoffigures

\newpage

\pagenumbering{arabic}

\section{Introduction}

The module guide document is providing to the reader the decomposition of 
\progname into smaller pieces to help the implementation phase.
Decomposing a system into modules is a commonly accepted approach to developing
software.  A module is a work assignment for a programmer or programming
team~\citep{ParnasEtAl1984}.  We advocate a decomposition
based on the principle of information hiding~\citep{Parnas1972a}.  This
principle supports design for change, because the ``secrets'' that each module
hides represent likely future changes.  Design for change is valuable in SC,
where modifications are frequent, especially during initial development as the
solution space is explored.  

Our design follows the rules laid out by \citet{ParnasEtAl1984}, as follows:
\begin{itemize}
\item System details that are likely to change independently should be the
  secrets of separate modules.
\item Each data structure is used in only one module.
\item Any other program that requires information stored in a module's data
  structures must obtain it by calling access programs belonging to that module.
\end{itemize}

After completing the first stage of the design, the Software Requirements
Specification (SRS), the Module Guide (MG) is developed~\citep{ParnasEtAl1984}.
The MG specifies the modular structure of the system and is intended to allow 
both designers and maintainers to easily identify the parts of the software. 
The potential readers of this document are as follows:

\begin{itemize}
\item New project members: This document can be a guide for a new project
  member to easily understand the overall structure and quickly find the
  relevant modules they are searching for.
\item Maintainers: The hierarchical structure of the module guide improves the
  maintainers' understanding when they need to make changes to the system. It 
  is important for a maintainer to update the relevant sections of the document
  after changes have been made.
\item Designers: Once the module guide has been written, it can be used to
  check for consistency, feasibility and flexibility. Designers can verify the
  system in various ways, such as consistency among modules, feasibility of the
  decomposition, and flexibility of the design.
\end{itemize}

The rest of the document is organized as follows. Section
\ref{SecChange} lists the anticipated and unlikely changes of the software
requirements. Section \ref{SecMH} summarizes the module decomposition that
was constructed according to the likely changes. Section \ref{SecConnection}
specifies the connections between the software requirements and the
modules. Section \ref{SecMD} gives a detailed description of the
modules. Section \ref{SecTM} includes two traceability matrices. One checks
the completeness of the design against the requirements provided in the SRS. 
The other shows the relation between anticipated changes and the modules. 
Section \ref{SecUse} describes the use relation between modules.

\section{Anticipated and Unlikely Changes} \label{SecChange}

This section lists possible changes to the system. According to the likeliness
of the change, the possible changes are classified into two
categories. Anticipated changes are listed in Section \ref{SecAchange}, and
unlikely changes are listed in Section \ref{SecUchange}.

\subsection{Anticipated Changes} \label{SecAchange}

Anticipated changes are the source of the information that is to be hidden
inside the modules. Ideally, changing one of the anticipated changes will only
require changing the one module that hides the associated decision. 
The approach adapted here is called design for change.

\begin{description}
\item[\refstepcounter{acnum} \actheacnum \label{acHardware}:] The specific
  hardware on which \progname software is running.
\item[\refstepcounter{acnum} \actheacnum \label{acFormat}:] The method of the
  user input (ex. command-line option vs. dialog box)
\item[\refstepcounter{acnum} \actheacnum \label{ac3d}:] The 3D coordinate space 
  fixed to a 2D motion in 3D space
\item[\refstepcounter{acnum} \actheacnum \label{acCoord}:] The format of the
  Lagrangian (2D vs 3D)
\item[\refstepcounter{acnum} \actheacnum \label{acLinear}:] The (linear)
  solver algorithm
\item[\refstepcounter{acnum} \actheacnum \label{acCont}:] The content of the
  simulation and/or output (eg. 2D plot instead of 3D)
\end{description}

\subsection{Unlikely Changes} \label{SecUchange}

The module design should be as general as possible. However, a general system 
is more complex. Sometimes this complexity is not necessary. Fixing some design
decisions at the system architecture stage can simplify the software design. If
these decision should later need to be changed, then many parts of the design
will potentially need to be modified. Hence, it is not intended that these
decisions will be changed.

\begin{description}
\item[\refstepcounter{ucnum} \uctheucnum \label{ucIO}:] Input/Output devices\\
  (Input: File and/or Keyboard, Output: File, Memory, and/or Screen)
\item[\refstepcounter{ucnum} \uctheucnum \label{ucGoal}:] The goal statements 
 of \progname (see SRS; extensions to SRS are still possible, but current goals
 should not change)
\item[\refstepcounter{ucnum} \uctheucnum \label{ucData}:] The data types 
 for the simulation and plot 
\end{description}

\section{Module Hierarchy} \label{SecMH}

This section provides an overview of the module design. Modules are summarized
in a hierarchy decomposed by secrets in Table \ref{TblMH}. The modules listed
below, which are leaves in the hierarchy tree, are the modules that will
actually be implemented.

\begin{description}
\item [\refstepcounter{mnum} \mthemnum \label{mHH}:] Hardware-Hiding Module 
(described in MG$\_$Hardware)
\item [\refstepcounter{mnum} \mthemnum \label{mHH}:] \progname Control Module 
(described in MG$\_$Control)
\item [\refstepcounter{mnum} \mthemnum \label{mHH}:] \progname GUI Module 
(described MG$\_$GUI)
\item [\refstepcounter{mnum} \mthemnum \label{mHH}:] User Input Parameters 
Module (described in MG$\_$InputFormat)
\item [\refstepcounter{mnum} \mthemnum \label{mHH}:] Lagrangian Module 
(described in MG$\_$LA)
\item [\refstepcounter{mnum} \mthemnum \label{mHH}:] Hamiltonian Module 
(described in MG$\_$HA)
\item [\refstepcounter{mnum} \mthemnum \label{mHH}:] Data Structure Module 
(described in MG$\_$DataStruct)
\item [\refstepcounter{mnum} \mthemnum \label{mHH}:] Generic GUI/Plot Module 
(described in MG$\_$GUIplot)
\item [\refstepcounter{mnum} \mthemnum \label{mHH}:] Generic Trajectory 
Simulation GUI Module (described in MG$\_$SIM)
\end{description}

\begin{table}[h!]
\centering
\begin{tabular}{p{0.3\textwidth} p{0.6\textwidth}}
\toprule
\textbf{Level 1} & \textbf{Level 2}\\
\midrule

{Hardware-Hiding Module} & ~ \\
\midrule
\multirow{7}{0.3\textwidth}{Behaviour-Hiding Module} & 
{\progname Control Module}\\
& \progname GUI Module\\
& User Input Parameters Module\\
& Data Structure Module\\
\midrule
\multirow{3}{0.3\textwidth}{Software Decision Module} & {Generic Trajectory 
Simulation GUI Module}\\
& Generic GUI/Plot Module\\
& Lagrangian Module\\
& Hamiltonian Module\\ 
\bottomrule

\end{tabular}
\caption{Module Hierarchy}
\label{TblMH}
\end{table}

\section{Connection Between Requirements and Design} \label{SecConnection}

The design of the system is intended to satisfy the requirements developed in
the SRS. In this stage, the system is decomposed into modules. The connection
between requirements and modules is listed in Table \ref{TblRT}.

\section{Module Decomposition} \label{SecMD}

Modules are decomposed according to the principle of ``information hiding''
proposed by \citet{ParnasEtAl1984}. The \emph{Secrets} field in a module
decomposition is a brief statement of the design decision hidden by the
module. The \emph{Services} field specifies \emph{what} the module will do
without documenting \emph{how} to do it. For each module, a suggestion for the
implementing software is given under the \emph{Implemented By} title. If the
entry is \emph{OS}, this means that the module is provided by the operating
system or by standard programming language libraries.  Also indicate if the
module will be implemented specifically for the software.

Only the leaf modules in the
hierarchy have to be implemented. If a dash (\emph{--}) is shown, this means
that the module is not a leaf and will not have to be implemented. Whether or
not this module is implemented depends on the programming language
selected.

\subsection{Hardware Hiding Modules (M1)}%(\mref{mHH})}

\begin{description}
\item[Secrets:]The data structure and algorithm used to implement the virtual
  hardware.
\item[Services:]Serves as a virtual hardware used by the rest of the
  system. This module provides the interface between the hardware and the
  software. So, the system can use it to display outputs or to accept inputs.
\item[Implemented By:] OS
\end{description}

\subsection{Behaviour-Hiding Module}

\begin{description}
\item[Secrets:]The contents of the required behaviours.
\item[Services:]Includes programs that provide externally visible behaviour of
  the system as specified in the software requirements specification (SRS)
  documents. This module serves as a communication layer between the
  hardware-hiding module and the software decision module. The programs in this
  module will need to change if there are changes in the SRS.
\item[Implemented By:] --
\end{description}

\subsubsection{\progname{}Control Module (M2)}
\label{MG_Control}
\begin{description}
\item[Secrets:] Execution flow of \progname{}. \wss{At the end of a sentence
    you want to just use progname without the parentheses.} 
\item[Services:] Calls the different modules in the appropriate order.
\item[Implemented By:] \progname{}
\end{description}

\subsubsection{\progname{}GUI (M3)}
\label{MG_GUI}
\begin{description}
\item[Secrets:] Methods to interact with user to make run \progname{}. \wss{Are
    you combining your GUI and your control?  You might want to consider the
    Model View Controller pattern.  This would mean that the view wouldn't
    actually control anything.}
\item[Services:] Serves as interface between the user and the software through 
the hardware by outputting calculation results and collecting user information 
(user inputs) for the calculation modules. 
\item[Implemented By:] \progname{}
\end{description}

\subsubsection{User Input Module (M4)}
\label{MG_InputFormat}
\begin{description}
\item[Secrets:] The format and structure of the input data.
\item[Services:] Loads, verifies and stores the input data into the appropriate 
data and object structure.
\item[Implemented By:] \progname{}
\end{description}

\subsection{Software Decision Module}

\begin{description}
\item[Secrets:] The design decision based on mathematical theorems, physical
  facts, or programming considerations. The secrets of this module are
  \emph{not} described in the SRS.
\item[Services:] Includes data structure and algorithms used in the system that
  do not provide direct interaction with the user. 
  % Changes in these modules are more likely to be motivated by a desire to
  % improve performance than by externally imposed changes.
\item[Implemented By:] --
\end{description}

\subsubsection{Lagrangian Module (M5)}
\label{MG_LA}
\begin{description}
\item[Secrets:] Lagrangian calculation/algorithm.
\item[Services:] Calculates the variation of Lagrangian for each pendulum in 
the chain (each element of the array).
\item[Implemented By:] \progname{}
\end{description}

\subsubsection{Hamiltonian Module (M6)}
\label{MG_HA}
\begin{description}
\item[Secrets:] Hamiltonian calculation/algorithm.
\item[Services:] Calculates the variation of Hamiltonian for each pendulum in 
the chain (each element of the array).
\item[Implemented By:] \progname{}
\end{description}

\subsubsection{Data Structure Module (M7)}
\label{MG_DataStruct}
\begin{description}
\item[Secrets:] Data format for an image.
\item[Services:] Provides convenient format to store, read and manipulate all 
elements (pixel) from an image.
\item[Implemented By:] Python library
\end{description}

\subsubsection{Generic GUI/Plot Module (M8)}
\label{MG_GUIGene}
\begin{description}
\item[Secrets:] Generic methods to interact with the user.
\item[Services:] Provides the generic interface methods such as buttons, 
windows, plotting, entry fields to interact with a user.
\item[Implemented By:] Python library
\end{description}

\subsubsection{Trajectory Simulation Module (M9)}
\label{MG_SMHSim}
\begin{description}
\item[Secrets:] Algorithm to simulate the plot trajectory of the chain 
of pendula.
\item[Services:] Simulates the chain of pendula in space using the Lagrangian
and the Hamiltonian.
\item[Implemented By:] \progname{}
\end{description}

\section{Traceability Matrix} \label{SecTM}

This section shows two traceability matrices: between the modules and the
requirements and between the modules and the anticipated changes.

% the table should use mref, the requirements should be named, use something
% like fref
\begin{table}[H]
\centering
\begin{tabular}{p{0.2\textwidth} p{0.6\textwidth}}
\toprule
\textbf{Req.} & \textbf{Modules}\\
\midrule
R1 & M2, M3, M4, M5, M6, M7, M8, M9\\%\mref{mHH}, \mref{mInput}, 
\mref{mParams}, \mref{mControl}\\
R2 & M2, M3, M4, M5, M6\\%\mref{mInput}, \mref{mParams}\\
R3 & M4, M5, M6, M7, M8, M9\\%\mref{mVerify}\\
R4 & M4, M5, M6, M7, M8, M9\\%\mref{mOutput}, \mref{mControl}\\
\bottomrule
\end{tabular}
\caption{Trace Between Requirements and Modules}
\label{TblRT}
\end{table}

\begin{table}[H]
\centering
\begin{tabular}{p{0.2\textwidth} p{0.6\textwidth}}
\toprule
\textbf{AC} & \textbf{Modules}\\
\midrule
\acref{acHardware} & M1, \mref{mHH}\\
\acref{acFormat} & M4, M5, M6, M7\\%\mref{mInput}\\
\acref{ac3d} & M2, M3, M8, M9\\%\mref{mParams}\\
\acref{acCoord} & M2, M3, M8, M9\\%\mref{mVerify}\\
\acref{acLinear} & M4, M5, M6, M7\\%\mref{mOutput}\\
\acref{acCont} & M4, M5, M6, M7\\%\mref{mVerifyOut}\\
\bottomrule
\end{tabular}
\caption{Trace Between Anticipated Changes and Modules}
\label{TblACT}
\end{table}

\section{Use Hierarchy Between Modules} \label{SecUse}

In this section, the uses hierarchy between modules is provided. 
\citet{Parnas1978} said of two programs A and B that A {\em uses} B if
correct execution of B may be necessary for A to complete the task described in
its specification. That is, A {\em uses} B if there exist situations in which
the correct functioning of A depends upon the availability of a correct
implementation of B.  Figure \ref{FigUH} illustrates the use relation between
the modules. It can be seen that the graph is a directed acyclic graph
(DAG). Each level of the hierarchy offers a testable and usable subset of the
system, and modules in the higher level of the hierarchy are essentially 
simpler because they use modules from the lower levels.

\begin{figure}[H]
\centering
\includegraphics[width=1.0\textwidth]{figure1.PNG}
\caption{Use hierarchy among modules}
\label{FigUH}
\end{figure}

\wss{The uses relation is confusing.  The input module should be ``used by'' 
  other modules, not using them. I would think the input module would be lower 
  in the hierarchy. I don't know why the trajectory simulation module uses the 
GUI module?  Maybe this will be clear to me (and you) when you write the MIS.}

\newpage

\bibliographystyle {plainnat}
\bibliography{../../../ReferenceMaterial/References}

\end{document}
