\documentclass[12pt, titlepage]{article}

\usepackage{amsmath, mathtools}

\usepackage[round]{natbib}
\usepackage{amsfonts}
\usepackage{amssymb}
\usepackage{graphicx}
\usepackage{colortbl}
\usepackage{xr}
\usepackage{hyperref}
\usepackage{longtable}
\usepackage{xfrac}
\usepackage{tabularx}
\usepackage{float}
\usepackage{siunitx}
\usepackage{booktabs}
\usepackage{multirow}
\usepackage[section]{placeins}
\usepackage{caption}
\usepackage{fullpage}

\hypersetup{
bookmarks=true,% show bookmarks bar?
colorlinks=true,% false: boxed links; true: colored links
linkcolor=red,% color of internal links (change box color with linkbordercolor)
citecolor=blue,% color of links to bibliography
filecolor=magenta,% color of file links
urlcolor=cyan% color of external links
}

\usepackage{array}

\externaldocument{../../SRS/SRS}

%% Comments

\usepackage{color}

\newif\ifcomments\commentsfalse

\ifcomments
\newcommand{\authornote}[3]{\textcolor{#1}{[#3 ---#2]}}
\newcommand{\todo}[1]{\textcolor{red}{[TODO: #1]}}
\else
\newcommand{\authornote}[3]{}
\newcommand{\todo}[1]{}
\fi

\newcommand{\wss}[1]{\authornote{blue}{SS}{#1}}
\newcommand{\spc}[1]{\authornote{magenta}{SP}{#1}}
%% Common Parts

\newcommand{\progname}{MPS } % PUT YOUR PROGRAM NAME HERE


\begin{document}

\title{CAS 741: Module Interface Specification\\[10pt]\Large Dynamical 
Systems: \progname}
\author{Karol Serkis\\\texttt{serkiskj@mcmaster.ca}\\GitHub:
\href{https://www.github.com/karolserkis}{karolserkis}}
\date{\today}

\maketitle

\pagenumbering{roman}

\section{Revision History}

\begin{tabularx}{\textwidth}{p{4cm}p{2cm}X}
\toprule {\bf Date} & {\bf Version} & {\bf Notes}\\
\midrule
November 26, 2018 & 1.0 &  First full draft for submission\\
December 17, 2018 & 2.0 & All GitHub issues and comments addressed \\
December 18, 2018 & 3.0 & MIS \& bibliography fixed, 
\& program name fixed \\
\bottomrule
\end{tabularx}

~\newpage

\section{Symbols, Abbreviations and Acronyms}

See SRS Documentation at 
\href{https://github.com/karolserkis/CAS-741-Pendula/blob/master/docs/SRS/SRS.pdf}
{CAS-741-Pendula SRS}

\wss{Also add any additional symbols, abbreviations or acronyms}\\
The symbols are listed in alphabetical order.\\

\renewcommand{\arraystretch}{1.2}
\begin{tabular}{l l} 
  \toprule		
  \textbf{symbol} & \textbf{description}\\
  \midrule 
  A & Assumption\\
  DD & Data Definition\\
  GD & General Definition\\
  GS & Goal Statement\\
  IM & Instance Model\\
  LC & Likely Change\\
  NF & Non-Functional Requirement\\
  MIS & Module Interface Specification\\
  PS & Physical System Description\\
  R & Requirement\\
  SRS & Software Requirements Specification\\
  T & Theoretical Model\\
  \bottomrule
\end{tabular}\\

\newpage

\tableofcontents

\newpage

\pagenumbering{arabic}

\section{Introduction}

The following document details the Module Interface Specifications for the 
\progname project
\wss{Fill in your project name and description}

Complementary documents include the System Requirement Specifications
and Module Guide. The full documentation and implementation can be
found at: \\\url{https://github.com/karolserkis/CAS-741-Pendula}.  
\wss{provide the url for your repo}

\section{Notation}

\wss{You should describe your notation.  You can use what is below as
  a starting point.}  \wss{You shouldn't keep my original comments in your
  version.  The SS comments for the references really look odd.}

The structure of the MIS for modules comes from \citet{HoffmanAndStrooper1995},
with the addition that template modules have been adapted from
\cite{GhezziEtAl2003}\wss{GhezziEtAl2003}.  The mathematical notation comes 
from Chapter 3 of \citet{HoffmanAndStrooper1995}.  
For instance, the symbol := is used for a multiple assignment statement and 
conditional rules follow the form 
$(c_1 \Rightarrow r_1 | c_2 \Rightarrow r_2 | ... | c_n \Rightarrow r_n )$.\\\\
(Karol) I am also using \href{https://proofwiki.org/wiki/Symbols:R}{this link}
for now.\\

The following table summarizes the primitive data types used by \progname and 
the symbols used in this document.
The choice of symbols was made to be consistent with calculus, ordinary
differentials (ODE), the Lagrangian, kinematics etc. The standard mathematical
spaces are used (e.g. $\mathbb{N}$, $\mathbb{Z}$, $\mathbb{R}$, etc.)

\begin{center}
\renewcommand{\arraystretch}{1.2}
\noindent 
\begin{tabular}{l l p{7.5cm}} 
\toprule 
\textbf{Data Type} & \textbf{Notation} & \textbf{Description}\\ 
\midrule
character & char & a single symbol or digit\\
integer & $\mathbb{Z}$ & a number without a fractional component in 
(-$\infty$, $\infty$) \\
natural number & $\mathbb{N}$ & a number without a fractional component in 
[1, $\infty$) \\
real & $\mathbb{R}$ & any number in (-$\infty$, $\infty$)\\
\bottomrule
\end{tabular} 
\end{center}

\noindent
The specification of \progname uses some derived data types: sequences, 
strings, and tuples. 
Sequences are lists filled with elements of the same data type. Strings
are sequences of characters. Tuples contain a list of values, potentially of
different types. In addition, \progname uses functions, which
are defined by the data types of their inputs and outputs. Local functions are
described by giving their type signature followed by their specification.

\section{Module Decomposition}

The following table is taken directly from the Module Guide document for 
this project.

\begin{table}[h!]
\centering
\begin{tabular}{p{0.3\textwidth} p{0.6\textwidth}}
\toprule
\textbf{Level 1} & \textbf{Level 2}\\
\midrule

{Hardware-Hiding} & ~ \\
\midrule

\multirow{7}{0.3\textwidth}{Behaviour-Hiding} & \progname Control Module\\
& \progname GUI Module\\
& User Input Parameters Module\\
& Data Structure Module\\
\midrule

\multirow{3}{0.3\textwidth}{Software Decision} & {Generic Trajectory Simulation
 GUI Module}\\
& Generic GUI/Plot Module\\
& Lagrangian Module\\
& Hamiltonian Module\\ 
\bottomrule

\end{tabular}
\caption{Module Hierarchy}
\label{TblMH}
\end{table}

\newpage

%%%%%%%%%%%%%%%%%%%%%%%%%%%%%%%%%%%%%%
\section{MIS of \progname{}Control Module (M2)} \label{Module-Ctrl}

\wss{You can reference SRS labels, such as R\ref{R_Inputs}.}

\wss{It is also possible to use \LaTeX for hypperlinks to 
external documents.}\\

\noindent Execution flow of \progname{}. Calls the different modules in 
the appropriate order.
The general steps of the module are:
\begin{enumerate}
\item Read in and verify mps\_input.
\item Calculate Lagrangian module, iterate over the whole time frame 
(use data structure).
\item Calculate Hamiltonian module, iterate over the whole time frame 
(use data structure).
\item Generate a Plot and Poincar\'{e} map
\item Return the trajectory plot output as per the output/simulation module 
specifications.
\end{enumerate}

\subsection{Module}

mps\_control

\subsection{Uses}

mps\_gui (M3, Section 7)\\
mps\_input (M4, Section 8)\\\\
Processing modules: \wss{what are processing modules?  I'm not sure about 
this terminology}\\
mps\_lagrang (M5, Section 9), mps\_hamil (M6, Section 10)\\
mps\_datastruct (M7, Section 11)\\
generic\_plot (M8, Section 12), mps\_trajplotsim (M9, Section 13) 

\subsection{Syntax}

\subsubsection{Exported Constants}

None

\subsubsection{Exported Access Programs}

\begin{center}
\begin{tabular}{p{2cm} p{4cm} p{4cm} p{2cm}}
\hline
\textbf{Name} & \textbf{In} & \textbf{Out} & \textbf{Exceptions} \\
\hline
run\_mps & int, str & None & None \\
\hline
\end{tabular}
\end{center}

\subsection{Semantics}

The \progname is designed to have the input and initialization controlled by 
the user directly through the GUI.
The \progname Control Module uses the events in the \progname GUI to use the 
processing modules in order.

\subsubsection{State Variables}

\begin{itemize}
	\item mps\_input\_init : init (see Section 8)
	\item mps\_datastruct\_init : init (see Section 11)
	\item mps\_lagrang : calc (see Section 9)
	\item mps\_hamil : calc (see Section 10)
\end{itemize}

\wss{init is not a type, either is calc?  How am I supposed to read this? State
  variables are specified as a name and a type.}

\subsubsection{Environment Variables}

\wss{This section is not necessary for all modules.  Its purpose is to capture
  when the module has external interaction with the environment, such as for a
  device driver, screen interface, keyboard, file, etc.}

\subsubsection{Assumptions}

\wss{Try to minimize assumptions and anticipate programmer errors via
  exceptions, but for practical purposes assumptions are 
sometimes appropriate.}

\subsubsection{Access Routine Semantics}

\noindent run\_mps(mps\_input) 
\wss{This doesn't match the int, str specified above?}:
\begin{itemize}
\item transition: The Multi Pendulum Simulation is initialized and the data 
structure through the modules
for calculation (Lagrangian \& Hamiltonian. The result is then passed to the 
plot/simulation/output modules.
\item output: None \wss{if appropriate} 
\item exception: None \wss{if appropriate} 
\end{itemize}

\wss{A module without environment variables or state variables is unlikely to
  have a state transition.  In this case a state transition can only occur if
  the module is changing the state of another module.}

\wss{Modules rarely have both a transition and an output.  In most cases you
  will have one or the other.}

\subsubsection{Local Functions}

\wss{As appropriate} \wss{These functions are for the purpose of specification.
  They are not necessarily something that is going to be implemented
  explicitly.  Even if they are implemented, they are not exported; they only
  have local scope.}

\newpage \wss{A new page between modules makes the document a more convenient
  reference document.}

%%%%%%%%%%%%%%%%%%%%%%%%%%%%%%%%%%%%%%
\section{MIS of \progname{}GUI (M3)} \label{Module} \wss{Use labels for
  cross-referencing}

\wss{You can reference SRS labels, such as R\ref{R_Inputs}.}

\wss{It is also possible to use \LaTeX for hyperlinks to external documents.}\\

Methods to interact with user to make run \progname{}. Serves as 
interface between the user and the software through 
the hardware by outputting calculation results and collecting user information 
(user inputs) for the calculation modules.

\subsection{Module}

mps\_gui

\subsection{Uses}

mps\_datastruct (M7, Section 11)\\
generic\_plot (M8, Section 12), mps\_trajplotsim (M9, Section 13) 

\subsection{Syntax}

\subsubsection{Exported Constants}

None

\subsubsection{Exported Access Programs}

\begin{center}
\begin{tabular}{p{2cm} p{4cm} p{4cm} p{2cm}}
\hline
\textbf{Name} & \textbf{In} & \textbf{Out} & \textbf{Exceptions} \\
\hline
mps\_input & int, str & - & - \\
gui\_trajsim & - & GUI object & - \\
gui\_plot & - & GUI object (Poincar\'{e}) & - \\
\hline
\end{tabular}
\end{center}

\subsection{Semantics}

The \progname process flow is initialized buy user through the GUI. 
The user triggered the events that start the selected processing step.

\subsubsection{State Variables}

None

\subsubsection{Environment Variables}

\noindent Plot{\_}TrajSim: GUI object\\
button{\_}Input: GUI object\\
button{\_}TrajSim: GUI object\\
button{\_}Poincar\'{e}: GUI object\\

\wss{More words to describe these environment variables would be helpful.}

\wss{This section is not necessary for all modules.  Its purpose is to capture
  when the module has external interaction with the environment, such as for a
  device driver, screen interface, keyboard, file, etc.}

\subsubsection{Assumptions}

\wss{Try to minimize assumptions and anticipate programmer errors via
exceptions, but for practical purposes assumptions are sometimes appropriate.}

\subsubsection{Access Routine Semantics}

\noindent mps\_input():
\begin{itemize}
\item transition: Trigger mps\_input when button\_Input pressed.
\item output: \wss{if appropriate} 
\item exception: WarnOutOfBounds (warn when mps\_input are invalid) 
\wss{if appropriate} 
\end{itemize}

\noindent gui\_trajsim():
\begin{itemize}
\item transition: Trigger gui\_trajsim when button\_TrajSim pressed.
\item output: \wss{if appropriate} 
\item exception: \wss{if appropriate} 
\end{itemize}

\noindent gui\_plot():
\begin{itemize}
\item transition: Trigger gui\_plot when button\_Poincar\'{e} pressed.
\item output: \wss{if appropriate} 
\item exception: \wss{if appropriate} 
\end{itemize}

\wss{A module without environment variables or state variables is unlikely to
  have a state transition.  In this case a state transition can only occur if
  the module is changing the state of another module.}

\wss{Modules rarely have both a transition and an output.  In most cases you
  will have one or the other.}

\subsubsection{Local Functions}

\wss{As appropriate} \wss{These functions are for the purpose of specification.
  They are not necessarily something that is going to be implemented
  explicitly.  Even if they are implemented, they are not exported; they only
  have local scope.}

%%%%%%%%%%%%%%%%%%%%%%%%%%%%%%%%%%%%%%
\section{MIS of User Input Module (M4)} \label{Module} \wss{Use labels for
  cross-referencing}

\wss{You can reference SRS labels, such as R\ref{R_Inputs}.}

\wss{It is also possible to use \LaTeX for hyperlinks to external documents.}\\

The format and structure of the input data. Loads, verifies and stores the 
input data into the appropriate data and object structure.

\subsection{Module}

mps\_input

\subsection{Uses}

mps\_gui (M3, Section 7)\\
mps\_datastruct (M7, Section 11) \wss{The modules in the uses section should
actually be used in the specification.  I don't see where these are actually 
invoked.}

\subsection{Syntax}

\subsubsection{Exported Constants}

None

\subsubsection{Exported Access Programs}

\begin{center}
\begin{tabular}{p{2cm} p{4cm} p{4cm} p{2cm}}
\hline
\textbf{Name} & \textbf{In} & \textbf{Out} & \textbf{Exceptions} \\
\hline
verify\_Input & int, str & - & - \\
data\_struct & int, str & - & - \\
gui\_trajsim & - & GUI object & - \\
gui\_plot & - & GUI object (Poincar\'{e}) & - \\
\hline
\end{tabular}
\end{center}

\subsection{Semantics}

\subsubsection{State Variables}

data : object \wss{What is this? object is not a type. 
What data are you storing?}

\subsubsection{Environment Variables}

\noindent Plot{\_}TrajSim: GUI object\\
\wss{Describe this enviro variable.  Are you taking the data fields from
  corresponding fields in the GUI object?}
\wss{You have really integrated the GUI into your design. It is better if you
 almost don't mention the GUI with the idea that the modules can be implemented
 by any interface.  Maybe a model view controller pattern would work in your
 case?}

\wss{This section is not necessary for all modules.  Its purpose is to capture
  when the module has external interaction with the environment, such as for a
  device driver, screen interface, keyboard, file, etc.}

\subsubsection{Assumptions}

\wss{Try to minimize assumptions and anticipate programmer errors via
 exceptions, but for practical purposes assumptions are sometimes appropriate.}

\subsubsection{Access Routine Semantics}

\noindent verify\_Input():
\begin{itemize}
\item transition: \wss{if appropriate} 
\item output: \wss{if appropriate} 
\item exception: Check of int or str value out of bounds: 
($\exists \mathbb{I}:I_{\text{calc}} \notin \mathbb{R})$ \& $\Rightarrow$
NotReal \wss{I have no idea how to read this.  Are you just checking the type?
  You don't need to do that.  Why are reals mentioned?} \\\wss{if appropriate}
\end{itemize}

\noindent data\_struct(mps\_input):
\begin{itemize}
\item transition: mps\_input is the user input that is entered into the GUI. 
The input is used to initialize the data structure. \wss{What is the data
    structure? The MIS is where you make decisions about this.}
\item output: \wss{if appropriate} 
\item exception: \wss{if appropriate} 
\end{itemize}

\noindent gui\_trajsim(mps\_datastruct, mps\_input):
\begin{itemize}
\item transition: mps\_datastruct is a module that takes the mps\_input and 
thus the program gui\_trajsim will be prepared for the Lagrangian and 
Hamiltonian calculations done by their corresponding modules.
\item output: \wss{if appropriate} 
\item exception: \wss{if appropriate} 
\end{itemize}

\noindent gui\_plot(mps\_datastruct, mps\_input):
\begin{itemize}
\item transition: mps\_datastruct is a module that takes the mps\_input and 
thus the program gui\_plot will be prepared for the Lagrangian and Hamiltonian 
calculations done by their corresponding modules.
\item output: \wss{if appropriate} 
\item exception: \wss{if appropriate} 
\end{itemize}

\wss{A module without environment variables or state variables is unlikely to
  have a state transition.  In this case a state transition can only occur if
  the module is changing the state of another module.}

\wss{Modules rarely have both a transition and an output.  In most cases you
  will have one or the other.}

\subsubsection{Local Functions}

\wss{As appropriate} \wss{These functions are for the purpose of specification.
  They are not necessarily something that is going to be implemented
  explicitly.  Even if they are implemented, they are not exported; they only
  have local scope.}
  
%%%%%%%%%%%%%%%%%%%%%%%%%%%%%%%%%%%%%%
\section{MIS of Lagrangian Module (M5)} \label{Module} \wss{Use labels for
  cross-referencing}

\wss{You can reference SRS labels, such as R\ref{R_Inputs}.}

\wss{It is also possible to use \LaTeX for hyperlinks to external documents.}\\

Lagrangian calculation/algorithm. Calculates the variation of Lagrangian for 
each pendulum in the chain (each element of the array).

\subsection{Module}

mps\_lagrang

\subsection{Uses}

mps\_input (M4, Section 8)\\\\
Processing modules:\\
mps\_lagrang (M5, Section 9)\\
mps\_datastruct (M7, Section 11)\\

\subsection{Syntax}

\subsubsection{Exported Constants}

None

\subsubsection{Exported Access Programs}

\begin{center}
\begin{tabular}{p{2cm} p{4cm} p{4cm} p{2cm}}
\hline
\textbf{Name} & \textbf{In} & \textbf{Out} & \textbf{Exceptions} \\
\hline
mps\_input  & int, str & - & - \\
calc\_lagrang & int, str & long, str & - \\
\hline
\end{tabular}
\end{center}

\subsection{Semantics}

\subsubsection{State Variables}

\subsubsection{Environment Variables}

\wss{This section is not necessary for all modules.  Its purpose is to capture
  when the module has external interaction with the environment, such as for a
  device driver, screen interface, keyboard, file, etc.}

\subsubsection{Assumptions}

\wss{Try to minimize assumptions and anticipate programmer errors via
 exceptions, but for practical purposes assumptions are sometimes appropriate.}

\subsubsection{Access Routine Semantics}

\noindent mps\_input():
\begin{itemize}
\item transition: mps\_input passed to calc\_lagrang (which in turn the module 
utilizes mps\_datastruct).
\item output: \wss{if appropriate} 
\item exception: WarnOutOfBounds (warn when mps\_input are invalid) 
\wss{if appropriate} 
\end{itemize}

\noindent calc\_lagrang(mps\_input):
\begin{itemize}
\item transition: Utilizing the mps\_datastruct the mps\_input is calculated 
and passed to the gui\_trajsim and gui\_plot \wss{if appropriate} 
\item output: To gui\_trajsim and gui\_plot \wss{if appropriate} 
\item exception: WarnOutOfBounds (warn when mps\_input are invalid) 
\wss{if appropriate} 
\end{itemize}

\wss{A module without environment variables or state variables is unlikely to
  have a state transition.  In this case a state transition can only occur if
  the module is changing the state of another module.}

\wss{Modules rarely have both a transition and an output.  In most cases you
  will have one or the other.}

\subsubsection{Local Functions}

\wss{As appropriate} \wss{These functions are for the purpose of specification.
  They are not necessarily something that is going to be implemented
  explicitly.  Even if they are implemented, they are not exported; they only
  have local scope.}

\wss{For the Lagrangian and Hamiltonian modules, I expected to see equations,
  like those from your SRS.}

%%%%%%%%%%%%%%%%%%%%%%%%%%%%%%%%%%%%%%
\section{MIS of Hamiltonian Module (M6)} \label{Module} \wss{Use labels for
  cross-referencing}

\wss{You can reference SRS labels, such as R\ref{R_Inputs}.}

\wss{It is also possible to use \LaTeX for hyperlinks to external documents.}\\

Hamiltonian calculation/algorithm. Calculates the variation of Hamiltonian for 
each pendulum in the chain (each element of the array).

\subsection{Module}

mps\_hamil

\subsection{Uses}

mps\_input (M4, Section 8)\\\\
Processing modules:\\
mps\_lagrang (M5, Section 9), mps\_hamil (M6, Section 10)\\
mps\_datastruct (M7, Section 11)\\

\subsection{Syntax}

\subsubsection{Exported Constants}

None

\subsubsection{Exported Access Programs}

\begin{center}
\begin{tabular}{p{2cm} p{4cm} p{4cm} p{2cm}}
\hline
\textbf{Name} & \textbf{In} & \textbf{Out} & \textbf{Exceptions} \\
\hline
mps\_input  & int, str & - & - \\
calc\_lagrang & int, str & long, str & - \\
calc\_hamil & long, str & long, str & - \\
\hline
\end{tabular}
\end{center}

\subsection{Semantics}

\subsubsection{State Variables}

\subsubsection{Environment Variables}

\wss{This section is not necessary for all modules.  Its purpose is to capture
  when the module has external interaction with the environment, such as for a
  device driver, screen interface, keyboard, file, etc.}

\subsubsection{Assumptions}

\wss{Try to minimize assumptions and anticipate programmer errors via
 exceptions, but for practical purposes assumptions are sometimes appropriate.}

\subsubsection{Access Routine Semantics}

\noindent mps\_input():
\begin{itemize}
\item transition: mps\_input passed to calc\_lagrang (which in turn the module 
utilizes mps\_datastruct).
\item output: \wss{if appropriate} 
\item exception: WarnOutOfBounds (warn when mps\_input are invalid) 
\wss{if appropriate} 
\end{itemize}

\noindent calc\_lagrang(mps\_input):
\begin{itemize}
\item transition: Utilizing the mps\_datastruct the mps\_input is calculated 
and passed to the gui\_trajsim and gui\_plot \wss{if appropriate} 
\item output: To gui\_trajsim and gui\_plot \wss{if appropriate} 
\item exception: WarnOutOfBounds (warn when mps\_input are invalid) 
\wss{if appropriate} 
\end{itemize}

\noindent calc\_hamil(mps\_input):
\begin{itemize}
\item transition: Utilizing the mps\_datastruct the mps\_input is calculated 
and passed to the gui\_trajsim and gui\_plot \wss{if appropriate} 
\item output: To gui\_trajsim and gui\_plot \wss{if appropriate} 
\item exception: WarnOutOfBounds (warn when mps\_input are invalid) 
\wss{if appropriate} 
\end{itemize}

\wss{A module without environment variables or state variables is unlikely to
  have a state transition.  In this case a state transition can only occur if
  the module is changing the state of another module.}

\wss{Modules rarely have both a transition and an output. In most cases you
  will have one or the other.}

\subsubsection{Local Functions}

\wss{As appropriate} \wss{These functions are for the purpose of specification.
  They are not necessarily something that is going to be implemented
  explicitly.  Even if they are implemented, they are not exported; they only
  have local scope.}

%%%%%%%%%%%%%%%%%%%%%%%%%%%%%%%%%%%%%%
\section{MIS of Data Structure Module (M7)} \label{Module} \wss{Use labels for
  cross-referencing}

\wss{You can reference SRS labels, such as R\ref{R_Inputs}.}

\wss{It is also possible to use \LaTeX for hyperlinks to external documents.}\\

Data format for an image. Provides convenient format to store, read and 
manipulate all elements (pixel) from an image.

\subsection{Module}

mps\_datastruct

\subsection{Uses}

mps\_gui (M3, Section 7)\\
mps\_input (M4, Section 8)\\\\
Processing modules:\\
mps\_lagrang (M5, Section 9), mps\_hamil (M6, Section 10)\\
mps\_datastruct (M7, Section 11)\\

\subsection{Syntax}

\subsubsection{Exported Constants}

None

\subsubsection{Exported Access Programs}

\begin{center}
\begin{tabular}{p{2cm} p{4cm} p{4cm} p{2cm}}
\hline
\textbf{Name} & \textbf{In} & \textbf{Out} & \textbf{Exceptions} \\
\hline
calc\_lagrang & int, str & long, str & - \\
calc\_hamil & long, str & long, str & - \\
mps\_gui & GUI object & - & - \\
\hline
\end{tabular}
\end{center}

\subsection{Semantics}

\subsubsection{State Variables}

\wss{the data structure doesn't have state variables?}

\subsubsection{Environment Variables}

\noindent Plot{\_}TrajSim: GUI object\\
Plot{\_}Poincar\'{e}: GUI object\\

\wss{Why does the data structure use gui objects?  As mentioned previously, the
GUI connects to too many modules. Most modules should be independent of the 
interface.}

\wss{This section is not necessary for all modules.  Its purpose is to capture
  when the module has external interaction with the environment, such as for a
  device driver, screen interface, keyboard, file, etc.}

\subsubsection{Assumptions}

\wss{Try to minimize assumptions and anticipate programmer errors via
 exceptions, but for practical purposes assumptions are sometimes appropriate.}

\subsubsection{Access Routine Semantics}

\noindent calc\_lagrang(mps\_input):
\begin{itemize}
\item transition: Utilizing the mps\_datastruct the mps\_input is calculated 
and passed to the gui\_trajsim and gui\_plot \wss{if appropriate} 
\item output: To gui\_trajsim and gui\_plot \wss{if appropriate} 
\item exception: WarnOutOfBounds (warn when mps\_input are invalid) 
\wss{if appropriate} 
\end{itemize}

\noindent calc\_hamil(mps\_input):
\begin{itemize}
\item transition: Utilizing the mps\_datastruct the mps\_input is calculated 
and passed to the gui\_trajsim and gui\_plot \wss{if appropriate} 
\item output: To gui\_trajsim and gui\_plot \wss{if appropriate} 
\item exception: WarnOutOfBounds (warn when mps\_input are invalid) 
\wss{if appropriate} 
\end{itemize}

\noindent mps\_gui(mps\_datastruct, gui\_trajsim, gui\_plot):
\begin{itemize}
\item transition: Utilizing the mps\_datastruct the mps\_input is calculated 
and passed to the gui\_trajsim and gui\_plot 
and thus the program gui\_trajsim will be prepared for the Lagrangian and 
Hamiltonian calculations done by their corresponding modules.
\item output: To gui\_trajsim and gui\_plot \wss{if appropriate} 
\item exception: WarnOutOfBounds (warn when mps\_input are invalid) 
\wss{if appropriate} 
\end{itemize}


\wss{A module without environment variables or state variables is unlikely to
  have a state transition.  In this case a state transition can only occur if
  the module is changing the state of another module.}

\wss{Modules rarely have both a transition and an output. In most cases you
  will have one or the other.}

\subsubsection{Local Functions}

\wss{As appropriate} \wss{These functions are for the purpose of specification.
  They are not necessarily something that is going to be implemented
  explicitly.  Even if they are implemented, they are not exported; they only
  have local scope.}

%%%%%%%%%%%%%%%%%%%%%%%%%%%%%%%%%%%%%%
\section{MIS of Generic GUI/Plot Module (M8)} \label{Module} 
\wss{Use labels for cross-referencing}

\wss{You can reference SRS labels, such as R\ref{R_Inputs}.}

\wss{It is also possible to use \LaTeX for hyperlinks to external documents.}\\

Generic methods to interact with the user. Provides the generic interface 
methods such as buttons, windows, plotting, entry fields to interact 
with a user.

\subsection{Module}

generic\_plot

\subsection{Uses}

Hardware-Hiding (M1)

\subsection{Syntax}

\subsubsection{Exported Constants}

None

\subsubsection{Exported Access Programs}

\begin{center}
\begin{tabular}{p{2cm} p{4cm} p{4cm} p{2cm}}
\hline
\textbf{Name} & \textbf{In} & \textbf{Out} & \textbf{Exceptions} \\
\hline
gui\_trajsim & GUI object & GUI object & - \\
gui\_plot & GUI object & GUI object & - \\
\hline
\end{tabular}
\end{center}

\subsection{Semantics}

\subsubsection{State Variables}

\subsubsection{Environment Variables}

\noindent Plot{\_}TrajSim: GUI object\\
Plot{\_}Poincar\'{e}: GUI object\\
\wss{This section is not necessary for all modules.  Its purpose is to capture
  when the module has external interaction with the environment, such as for a
  device driver, screen interface, keyboard, file, etc.}

\subsubsection{Assumptions}

\wss{Try to minimize assumptions and anticipate programmer errors via
 exceptions, but for practical purposes assumptions are sometimes appropriate.}

\subsubsection{Access Routine Semantics}

\noindent gui\_trajsim():
\begin{itemize}
\item transition: \wss{if appropriate} 
\item output: To mps\_gui(mps\_datastruct, gui\_trajsim, gui\_plot) or the 
generic\_plot module \wss{if appropriate} 
\item exception: WarnOutOfBounds (warn when gui\_trajsim are invalid) 
\wss{if appropriate} 
\end{itemize}

\noindent gui\_plot(mps\_datastruct, mps\_input):
\begin{itemize}
\item transition: mps\_datastruct is a module that takes the mps\_input and 
thus the program gui\_plot will be prepared for the Lagrangian and Hamiltonian 
calculations done by their corresponding modules.
\item output: \wss{if appropriate}  \wss{if it isn't appropriate, you should
    remove this field.}
\item exception: \wss{if appropriate} 
\end{itemize}

\wss{A module without environment variables or state variables is unlikely to
  have a state transition.  In this case a state transition can only occur if
  the module is changing the state of another module.}

\wss{Modules rarely have both a transition and an output. In most cases you
  will have one or the other.}

\subsubsection{Local Functions}

\wss{As appropriate} \wss{These functions are for the purpose of specification.
  They are not necessarily something that is going to be implemented
  explicitly.  Even if they are implemented, they are not exported; they only
  have local scope.}

%%%%%%%%%%%%%%%%%%%%%%%%%%%%%%%%%%%%%%
\section{MIS of Trajectory Simulation Module (M9)} \label{Module} 
\wss{Use labels for cross-referencing}

\wss{You can reference SRS labels, such as R\ref{R_Inputs}.}

\wss{It is also possible to use \LaTeX for hyperlinks to external documents.}\\

Algorithm to simulate the plot trajectory of the chain of pendula.
Simulates the chain of pendula in space using the Lagrangian
and the Hamiltonian.

\subsection{Module}

mps\_trajplotsim

\subsection{Uses}

mps\_gui (M3, Section 7)\\
mps\_input (M4, Section 8)\\\\
Processing modules:\\
mps\_lagrang (M5, Section 9), mps\_hamil (M6, Section 10)\\

\subsection{Syntax}

\subsubsection{Exported Constants}

None

\subsubsection{Exported Access Programs}

\begin{center}
\begin{tabular}{p{2cm} p{4cm} p{4cm} p{2cm}}
\hline
\textbf{Name} & \textbf{In} & \textbf{Out} & \textbf{Exceptions} \\
\hline
MPS\_trajsim & GUI object & GUI object & WarnOutOfBounds \\
\hline
\end{tabular}
\end{center}

\subsection{Semantics}

\subsubsection{State Variables}

data : object

\subsubsection{Environment Variables}

\noindent Plot{\_}TrajSim: GUI object\\

\wss{This section is not necessary for all modules.  Its purpose is to capture
  when the module has external interaction with the environment, such as for a
  device driver, screen interface, keyboard, file, etc.}

\subsubsection{Assumptions}

\wss{Try to minimize assumptions and anticipate programmer errors via
  exceptions, but for practical purposes assumptions are sometimes appropriate.}

\subsubsection{Access Routine Semantics}

\noindent MPS\_trajsim():
\begin{itemize}
\item transition: 
\begin{enumerate}
\item Read in and verify mps\_input.
\item Calculate Lagrangian module calc\_lagrang, iterate over the whole 
time frame (use data structure).
\item Calculate Hamiltonian module calc\_hamil, iterate over the whole 
time frame (use data structure).
\item Return the trajectory plot output as per the output/simulation module 
specifications.
\end{enumerate}
\item output: \wss{if appropriate} 
\item exception: WarnOutOfBounds (warn when mps\_input or calculations of the 
data structure are invalid) \wss{if appropriate} 
\end{itemize}

\wss{A module without environment variables or state variables is unlikely to
  have a state transition.  In this case a state transition can only occur if
  the module is changing the state of another module.}

\wss{Modules rarely have both a transition and an output.  In most cases you
  will have one or the other.}

\subsubsection{Local Functions}

\wss{As appropriate} \wss{These functions are for the purpose of specification.
  They are not necessarily something that is going to be implemented
  explicitly.  Even if they are implemented, they are not exported; they only
  have local scope.}

\newpage

%\section*{References}
\subsection*{References}\label{ssec:ref} \wss{You shouldn't do this manually,
  especially since you use BibTeX in the next section.  The new references can
  be added to your .bib file and everything will be generated automatically.}
\begin{itemize}
\item{[1]} Dynamics of multiple pendula 
\\\url{http://wmii.uwm.edu.pl/~doliwa/IS-2012/Szuminski-2012-Olsztyn.pdf}
\item{[2]} Pendulum \\\url{https://en.wikipedia.org/wiki/Pendulum}
\item{[3]} Pendulum (mathematics)
\\\url{https://en.wikipedia.org/wiki/Pendulum_(mathematics)}
\item{[4]} Double Pendulum
\\\url{https://en.wikipedia.org/wiki/Double_pendulum}\item{[4]}
Differential-Algebraic Equations by Taylor Series
\\\url{http://www.cas.mcmaster.ca/~nedialk/daets/}
\item{[5]} Multi-body Lagrangian Simulations
\\\url{https://www.youtube.com/channel/UCCuLchOx0W0yoNE9KOCYlVQ}
\item{[6]} The double pendulum: Lagrangian formulation
\\\url{https://diego.assencio.com/?index=1500c66ae7ab27bb0106467c68feebc6}
\item{[7]} Poincaré map
\\\url{https://en.wikipedia.org/wiki/Poincar%C3%A9_map}
\item{[8]} D. L. Parnas, ``On the criteria to be used in decomposing systems 
into modules'', Comm. ACM, vol. 15, pp. 1053-1058, December 1972.
\item{[9]} D. Parnas, P. Clement, and D. M. Weiss, ``The modular structure of 
complex systems'', in International Conference on Software Engineering, 
pp. 408-419, 1984.
\item{[10]} D. L. Parnas, ``Designing software for ease of extension and 
contraction,'' in ICSE '78: Proceedings of the 3rd international conference on 
Software engineering, \\
(Piscataway, NJ, USA), pp. 264-277, IEEE Press, 1978.
\item{[11]} Smith and Lai(2005); Smith et al. (2007). \\
See bib file that doesn't work for me for full citation.
\end{itemize}

\bibliographystyle {plainnat}
\bibliography{../../../ReferenceMaterial/References}

\newpage

\section{Appendix} \label{Appendix}

\wss{Extra information if required}

\end{document}
